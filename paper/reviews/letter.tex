\documentclass[a4paper]{article}

%\usepackage[active]{srcltx}
\usepackage[T1]{fontenc}
\usepackage{amssymb,amsmath}
\usepackage{dsfont}
\usepackage{url}
%\newtheorem{thm}{Theorem}[section]
%\newtheorem{prop}[thm]{Proposition}
%\newtheorem{lemma}[thm]{Lemma}
%\newtheorem{cor}[thm]{Corollary}
%\theoremstyle{definition}
%\newtheorem{dfn}{Definition}[section]
%\newtheorem{cdtn}{Condition}[section]
%\newtheorem{exm}{Example}[section]
%\theoremstyle{remark}
%\newtheorem{rmk}{Remark}[section]

\title{Answer to the Referee's and the Board Member's report}
\author{P. Alquier, K. Meziani and G. Peyr\'e}
\date{}

\begin{document}

\maketitle

First, we would like to thank both Referees and the Board Member
for their careful reading of the paper and their useful comments.
We address the points raised in the reports of the two Referees
in details in this letter.

\section*{Answer to the first Referee}

We thank the first Referee for his positive comments on the paper.
The misprint mentionned was corrected in this new version of the
paper. (Katia, je te laisse corriger cette coquille dans le papier?)

\section*{Answer to the second Referee}

We thank the first Referee for his positive comments.
Regardind the 4 issues raised by the Referee:
\begin{enumerate}
 \item (Katia, a toi de completer)??
 \item To be fair, we read carefuly the reference provided by the Referee
 [G. M. D'Ariano, Measuring
Quantum States, in Quantum Optics and Spectroscopy of Solids, ed. by
T. Hakioglu and A. S. Shumovsky, (Kluwer Academic Publisher, Amsterdam
1997), page 175] and did not find any rate of convergence that can
be compared to our results (Theorems 2.1 and 2.2). In [D'Ariano, 1997]
it is proved that it is possible to estimate each coefficient of the density
matrix, denoted $\rho_{j,k}$ in our paper, and thanks to the central limit theorem,
they prove that the order of $\hat{\rho}_{j,k}-\rho_{j,k}$ is $1/\sqrt{n}$
where $n$ is the number of observations (note that when we also estimate the
order of $\hat{\rho}_{j,k}-\rho_{j,k}$ in our paper (Lemma 3), but using
Hoeffding's inequality instead of the central limit theorem).

However, our convergence results concern the norm
$$ \|\hat{\rho} - \rho \|^2 = \sum_{j=0}^{\infty} \sum_{k=0}^{\infty}
                            |\hat{\rho}_{j,k}-\rho_{j,k}|^2, $$
so to upper-bound $\hat{\rho}_{j,k}-\rho_{j,k}$ by $1/\sqrt{n}$: it leads
to a trivial $+\infty$ upper bound on $ \|\hat{\rho} - \rho \|^2 $.

Our thresholding method allows to build an estimator of the density
matrix $\tilde{\rho}$ such that
$$ \|\tilde{\rho} - \rho \|^2 \xrightarrow[n\rightarrow\infty]{} 0,$$
with rates given by Theorems 2.1 and 2.2. We did not find such a result
in [D'Ariano, 1997]. Moreover, it is to be noted that [D'Ariano, 1997]
claims that {\it the choice of the $k$-cutoff (...) should be done
carefully as a function of the number of data}. This is perfectly correct,
remark that our procedure is actually an implementation of this idea: we
threshold the estimators $\hat{\rho}_{j,k}$ adaptively to the number of
data {\it and} the regularity of the density matrix.

 \item The Referee is right: using only $n \leq 10^6$ observations
 could lead the reader to the erroneous conclusion that our method is
 not able to deal with larger datasets. So, we included simulations with
 $n=10^8$ observations, as recommended by the Referee.
 The computation time of the algorithm scales linearly with the number $n$ of samples, which is the standard scaling for 
 such a method. We remind the reviewer that our Matlab code is accessible online from the address: \\
 \url{https://www.ceremade.dauphine.fr/~peyre/codes/}
 
 
 \item The caption of the figure has been updated as suggested by the reviewer. 
\end{enumerate}


\end{document}
